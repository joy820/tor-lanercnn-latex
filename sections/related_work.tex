


\section{Related Work}

\paragraph{Motion Forecasting}
Traditional methods use hand-crafted features and rules based on human knowledge 
to model interactions and constraints in motion forecasting
% \cite{choi2013understanding,choi2012unified,deo2018would,helbing1995social,mehran2009abnormal,yamaguchi2011you,weichiuplay}, which are sometimes oversimplified and not scalable. 
% \cite{choi2013understanding,helbing1995social,mehran2009abnormal,yamaguchi2011you,weichiuplay}, which are sometimes oversimplified and not scalable. 
\cite{choi2013understanding,helbing1995social,mehran2009abnormal,yamaguchi2011you}, which are sometimes oversimplified and not scalable. 
Recently, learning-based approaches employ the deep learning and
significantly outperform traditional ones.
Given the actors and the scene, a deep
forecasting model first needs to design a format to encode the information.
To do so, previous methods \cite{precog,chauffeurnet,covernet} often rasterize
the trajectories of
actors into a Birds-Eye-View (BEV) image, with different channels representing
different observation timesteps, and then apply a CNN and RoI pooling
% \cite{fasterrcnn,maskrcnn} to extract actor features. 
\cite{fasterrcnn} to extract actor features. 
Maps can be encoded similarly
\cite{nmp,dsd,intentnet,chauffeurnet,mfp}. 
However, the square receptive fields of a CNN may not be efficient to
encode actor movements \cite{lgn}, which are typically long curves.
Moreover, the map rasterization may lose useful information like lane topologies. 
RNNs are an alternative way to encode actor kinematic information
\cite{matf,mfp,vectornet,tnt,socialgan,sociallstm} compactly and efficiently.
Recently, VectorNet \cite{vectornet} and LaneGCN \cite{lgn} generalized such compact encodings to map representations. 
VectorNet treats a map
as a collection of polylines and encodes them with a RNN, while LaneGCN builds a
graph of lanes and conducts convolutions over it. Different from all these
work, we encode both actors and maps in an unified graph representation, which
is more structured and powerful.


Modeling interactions among actors is also critical for
a multi-agent system. Pioneering learning-based work design a social-pooling mechanism
\cite{sociallstm,socialgan} to aggregate the information from nearby actors.
However, such a 
pooling operation may potentially lose
actor-specific information. To address this,
attention-mechanism \cite{sophie,socialatt,carnet,sun2019relational} or GNN-based
methods \cite{dsd,interacttransformer,lgn,spagnn,precog,mfp,ilvm,vectornet} build actor
interaction graphs (usually fully-connected with all actors or k-nearest
neighbors based),
and perform attention or message passing to update actor features.
Social convolutional pooling \cite{matf,socialconvpool,pip} has also been explored, which maintains the spatial
distribution of actors. 
However, most of these work do not explicitly consider map structures,
which largely affects interactions among actors in reality. 

To generate predicted trajectories, many works sample multi-modal futures under a conditional variational auto-encoder (CVAE) framework
\cite{desire,r2p2,mfp,precog,ilvm}, or with a multi-head/mode regressor
\cite{lgn,cui2019multimodal,mercat2020multi,mangalam2020not,choi2019drogon}.
Others output discrete sets of
trajectory samples \cite{dsd,covernet,multipath}. Recently, TNT \cite{tnt}
concurrently and independently designs a similar output parameterization as ours
where lanes are used as priors for the forecasting. 
In addition to this, we also contribute a novel graph representation and a powerful
architecture which significantly outperforms their results.



\paragraph{Graph Neural Networks}
Relying on operators like graph convolution and message passing, graph neural
networks (GNNs) and their variants
% \cite{scarselli2008graph,bruna2013spectral,li2015gated,kipf2016semi,hamilton2017inductive,liao2019lanczosnet}
\cite{scarselli2008graph,bruna2013spectral,li2015gated,kipf2016semi,hamilton2017inductive,ying2018hierarchical,gao2019graph}
generalize deep learning on regular graphs like grids to ones with irregular
topologies, and have achieved great successes in 
% They have achieved great successes in learning useful graph representations for
% various tasks \cite{monti2017geometric,qi20173d,teney2017graph,li2017situation,garcia2018few}.
various tasks \cite{monti2017geometric,teney2017graph,li2017situation,garcia2018few}.
We draw inspiration from the general concept ``ego-graph'' and propose
\textit{LaneRoI}, which is specially designed for lane graphs and captures both
the local map topologies and motions of an actor.
Moreover, to capture interactions among actors, we further propose an
interaction module which effectively communicates information among
\textit{LaneRoI} graphs.

